\newglossaryentry{bci}{
    type=\acronymtype,
    name={BCI},
    description={Interfaz cerebro-computadora (\emph{brain-computer interface})},
    first={interfaz cerebro-computadora (BCI, \emph{brain-computer interface})},
    plural={BCI},
    firstplural={interfaces cerebro-computadora (BCI, \emph{brain-computer interfaces})},
}

\newglossaryentry{ise}{
    type=\acronymtype,
    name={ISE},
    description={Intregral del error al cuadrado (\emph{integral square error})},
    first={intregral del error al cuadrado (ISE, \emph{integral square error})},
}

\newglossaryentry{itse}{
    type=\acronymtype,
    name={ITSE},
    description={Intregral del tiempo multiplicado por error al cuadrado (\emph{integral time square error})},
    first={intregral del tiempo multiplicado por el error al cuadrado (ITSE, \emph{integral square error})},
}

\newglossaryentry{iae}{
    type=\acronymtype,
    name={ITSE},
    description={Intregral del error absoluto (\emph{integral absolute error})},
    first={intregral del tiempo multiplicado por el error absoluto (IAE, \emph{integral absolute error})},
}

\newglossaryentry{itae}{
    type=\acronymtype,
    name={ITAE},
    description={Intregral del tiempo multiplicado por el error absoluto  (\emph{integral time absolute error})},
    first={intregral del tiempo multiplicado por el error absoluto (ITAE, \emph{integral time absolute error})},
}

\newglossaryentry{daq}{
    type=\acronymtype,
    name={DAQ},
    description={adquisición de datos (\emph{data acquisition})},
    first={adquisición de datos (DAQ, \emph{data acquisition})},
}

\newglossaryentry{os}{
    type=\acronymtype,
    name={OS},
    description={Sistema operativo (\emph{operating system})},
    first={sistema operativo (OS, \emph{operating system})},
}

\newglossaryentry{ram}{
    type=\acronymtype,
    name={RAM},
    description={Memoria de acceso aleatorio (\emph{random access memory})},
    first={memoria de acceso aleatorio (RAM, \emph{random access memory})},
}

\newglossaryentry{vi}{
    type=\acronymtype,
    name={VI},
    description={Instrumento virtual (\emph{virtual instrument})},
    first={instrumento virtual (VI, \emph{virtual instrument})},
    plural={VI},
    firstplural={instrumentos virtuales (VI, \emph{virtual instruments})},
}

\newglossaryentry{subvi}{
    type=\acronymtype,
    name={subVI},
    description={Sub-instrumento virtual (\emph{sub virtual instrument})},
    first={sub-instrumento virtual (subVI, \emph{sub virtual instrument})},
    plural={VI},
    firstplural={sub-instrumentos virtuales (subVI, \emph{sub virtual instruments})},
}

\newglossaryentry{fsm}{
    type=\acronymtype,
    name={FSM},
    description={Máquina de estados finitos (\emph{finite-state machine})},
    first={máquina de estados finitos (FSM, \emph{finite-state machine})},
    plural={FSM},
    firstplural={máquinas de estados finitos (FSM, \emph{finite-state machines})},
}

\newglossaryentry{eft}{
    type=\acronymtype,
    name={EFT},
    description={Extremadamente plana (\emph{extremely flat-top})},
    first={extremadamente plana (EFT, \emph{extremely flat-top})},
    plural={EFT},
    firstplural={extremadamente planas (EFT, \emph{extremely flat-tops})},
}

\newglossaryentry{rna}{
    type=\acronymtype,
    name={ANN},
    description={Red neuronal artificial (\emph{artificial neural network})},
    first={red neuronal artificial (ANN, \emph{artificial neural network})},
    plural={ANN},
    firstplural={redes neuronales artificiales (ANN, \emph{artificial neural networks})},
}

\newglossaryentry{ft}{
    type=\acronymtype,
    name={FT},
    description={Transformada de Fourier (\emph{Fourier transform})},
    first={transformada de Fourier (FT, \emph{Fourier transform})},
    plural={FT},
    descriptionplural={transformadas de Fourier},
    firstplural={transformadas de Fourier (FT, \emph{Fourier transforms})}
}

\newglossaryentry{fft}{
    type=\acronymtype,
    name={FFT},
    description={Transformada rápida de Fourier (\emph{fast Fourier transform})},
    first={transformada rápida de Fourier (FFT, \emph{fast Fourier transform})},
    plural={FFT},
    descriptionplural={transformadas rápidas de Fourier},
    firstplural={transformadas rápidas de Fourier (FFT, \emph{Fourier transforms})}
}

\newglossaryentry{mra}{
    type=\acronymtype,
    name={MRA},
    description={Análisis multiresolución (\emph{multiresolution analysis})},
    first={análisis multiresolución (MRA, \emph{multiresolution analysis})},
    plural={MRA},
    descriptionplural={análisis multiresolución},
    firstplural={análisis multiresolución (MRA, \emph{multiresolution analysis})},
} 

\newglossaryentry{af}{
    type=\acronymtype,
    name={AF},
    text={AF},
    description={Función atómica (\emph{atomic function})},
    first={función atómica (AF, \emph{atomic function})},
    plural={AF},
    descriptionplural={funciones atómicas},
    firstplural={funciones atómicas (AF, \emph{atomic functions})}
} 


\newglossaryentry{rbf}{
    type=\acronymtype,
    name={RBF},
    description={Función de base radial (\emph{radial basis function})},
    first={función de base radial (RBF, \emph{radial basis functions})},
    plural={RBF},
    descriptionplural={funciones de base radial},
    firstplural={funciones de base radial (RBF, \emph{radial basis functions})}
}

\newglossaryentry{gdl}{
    type=\acronymtype,
    name={GDL},
    description={Grados de libertad},
    first={grados de libertad (GDL)},
    %first={grados de libertad (GDL)\glsadd{gdlp}},
    % see=[Glosario:]{gdlp}
}

\newglossaryentry{im}{
    type=\acronymtype,
    name={IM},
    description={Movimiento imaginario (\emph{imaginary movement})},
    longplural={movimientos imaginarios},
    plural={IM},
    first={movimiento imaginario (IM, \emph{imaginary movement})},
    firstplural={movimientos imaginarios (IM, \emph{imaginary movements})},
}

\newglossaryentry{matlab}{
    type=\acronymtype,
    name={MATLAB},
    description={Laboratorio de matrices (\emph{Matrices Laboratory})},
    first={laboratorio de matrices (MATLAB, \emph{Matrices Laboratory})},
}

\newglossaryentry{wnet}{
    type=\acronymtype,
    name={WaveNet},
    description={Red neuronal \emph{wavelet}},
    plural={WaveNet},
    longplural={redes neuronales \emph{wavelet}},
    first={red neuronal \emph{wavelet} (WaveNet)},
    firstplural={redes neuronales \emph{wavelet} (WaveNet)},
}

\newglossaryentry{wnetiir}{
    type=\acronymtype,
    name={WaveNet-IIR},
    description={Red neuronal \emph{wavelet} con filtros IIR},
    plural={WaveNet-IIR},
    longplural={redes neuronales \emph{wavelet} con filtros IIR},
    first={red neuronal \emph{wavelet} con filtros IIR (WaveNet-IIR)},
    firstplural={redes neuronales \emph{wavelet} con filtros IIR (WaveNet-IIR)},
}

\newglossaryentry{pmr}{
    type=\acronymtype,
    name={PMR},
    description={Proporcional multiresolución (\emph{proportional multiresolution})},
    plural={PMR},
    longplural={proprocionales multiresolución},
    first={proporcional multiresolución (PMR, \emph{proportional multiresolution})},
}

\newglossaryentry{dft}{
    type=\acronymtype,
    name={DFT},
    description={Transformada discreta de Fourier (\emph{discrete Fourier transform})},
    plural={DFT},
    longplural={transformadas discretas de Forier},
    first={transformada discreta de Fourier (DFT, \emph{discrete Fourier transform})},
}

\newglossaryentry{dwt}{
    type=\acronymtype,
    name={DWT},
    description={Transformada \emph{wavelet} discreta (\emph{discrete wavelet transform})},
    plural={DWT},
    longplural={transformadas \emph{wavelets} discretas},
    first={transformada \emph{wavelet} discreta (DWT, \emph{discrete wavelet transform})},
}

\newglossaryentry{cwt}{
    type=\acronymtype,
    name={CWT},
    description={Transformada \emph{wavelet} continua (\emph{continue wavelet transform})},
    plural={CWT},
    longplural={transformadas \emph{wavelet} continuas},
    first={transformada \emph{wavelet} continua (CWT, \emph{continue wavelet transform})},
}

\newglossaryentry{cnn}{
    type=\acronymtype,
    name={CNN},
    description={Redes neuronales convolucionales (\emph{convolutional neural networks})},
    plural={CNN},
    longplural={redes neuronales convolucionales},
    first={red neuronal convolucional (CNN, \emph{convolutional neural network})},
}

\newglossaryentry{rnn}{
    type=\acronymtype,
    name={RNN},
    description={Redes neuronales recurrentes (\emph{recurrent neural networks})},
    plural={RNN},
    longplural={redes neuronales recurrentes},
    first={red neuronal recurrente (RNN, \emph{recurrent neural network})},
}

\newglossaryentry{lstm}{
    type=\acronymtype,
    name={LSTM},
    description={Memoria a corto y largo plazo (\emph{long short-term memory})},
    plural={LSTM},
    longplural={memorias a corto y largo plazo},
    first={memoria a corto y largo plazo (LSTM, \emph{long short-term memory})},
}

\newglossaryentry{iir}{
    type=\acronymtype,
    name={IIR},
    description={Respuesta infinita al impulso (\emph{infinite impulse response})},
    plural={DWT},
    longplural={respuesta infinita al impulso},
    first={respuesta infinita al impulso (IIR, \emph{infinite impulse response})},
}

\newglossaryentry{lms}{
    type=\acronymtype,
    name={LMS},
    description={Mínimos cuadrados medios (\emph{least mean square})},
    plural={LMS},
    first={mínimos cuadrados medios (LMS, \emph{least mean square})},
}

\newglossaryentry{fir}{
    type=\acronymtype,
    name={FIR},
    description={Respuesta finita al impulso (\emph{finite impulse response})},
    plural={FIR},
    longplural={respuesta finita al impulso},
    first={respuesta finita al impulso (FIR, \emph{finite impulse response})},
}

\newglossaryentry{fam}{
    type=\acronymtype,
    name={FAM},
    description={Memoria de asociación difusa (\emph{fuzzy associative memory})},
    plural={FAM},
    longplural={memoria de asociación difusa},
    first={memoria de asociación difusa (FAM, \emph{fuzzy associative memory})},
}

\newglossaryentry{siso}{
    type=\acronymtype,
    name={SISO},
    description={Una entrada y una salida (\emph{single-input and single-output)}},
    first={una entrada y una salida (SISO, \emph{single-input and single-output)}},
}

\newglossaryentry{rasp}{
    type=\acronymtype,
    name={RASP},
    description={Función racionales con polos de segundo orden (\emph{Rational function with Second-order Poles})},
    first={función racional con polos de segundo orden (RASP, \emph{Rational function with Second-order Poles})},
    plural={RASP},
    descriptionplural={funciones racionales con polos de segundo orden},
    firstplural={funciones racionales con polos de segundo orden (RASP, \emph{Rational functions with Second-order Poles})}
}

\newglossaryentry{polywog}{
    type=\acronymtype,
    name={POLYWOG},
    description={Ventana polinómica Gaussiana (\emph{POLYnominal WinOwed with Gaussian})},
    first={ventana polinómica Gaussiana (POLYWOG, \emph{POLYnominal WinOwed with Gaussian})}
    plural={POLYWOG},
    firstplural={ventanas polinómicas Gaussianas (POLYWOG, \emph{POLYnominal WinOwed with Gaussian})}
}

\newglossaryentry{mimo}{
    type=\acronymtype,
    name={MIMO},
    plural={MIMO},
    description={Múltiples entradas y múltiples salidas (\emph{multi-inputs and multi-outputs)}},
    first={múltiples entradas y múltiples salidas (MIMO, \emph{multi-inputs and multi-outputs)}},
}

\newglossaryentry{mlp}{
    type=\acronymtype,
    name={MLP},
    description={Perceptrón multicapa (\emph{multilayer perceptron})},
    first={perceptrón multicapa (MLP, \emph{multilayer perceptron})},
    plural={MLP},
    firstplural={perceptrones multicapa (MLP, \emph{multilayer perceptrons})},
}

\newglossaryentry{wf}{
    type=\acronymtype,
    name={WF},
    description={Función ventana (\emph{window function})},
    first={función ventana (WF, \emph{window function})},
    plural={WF},
    descriptionplural={funciones ventana},
    firstplural={funciones ventana (WF, \emph{window functions})}
}

\newglossaryentry{awf}{
    type=\acronymtype,
    name={AWF},
    description={Función de ventaneo adaptable (\emph{adaptive window function})},
    first={función de ventaneo adaptable (AWF, \emph{adaptive window function})},
    plural={AWF},
    descriptionplural={funciones de ventaneo adaptables},
    firstplural={funciones de ventaneo adaptables (AWF, \emph{adaptive window functions})}
}

\newglossaryentry{svm}{
    type=\acronymtype,
    name={SVM},
    description={Máquina de soporte vectorial (\emph{support vector machine})},
    first={máquina de soporte vectorial (SVM, \emph{support vector machine})},
    plural={SVM},
    descriptionplural={máquinas de soporte vectorial},
    firstplural={máquinas de soporte vectorial (SVM, \emph{support vector machines})}
}

\newglossaryentry{apc}{
    type=\acronymtype,
    name={APC},
    description={Adquisición, procesamiento y clasificación},
    first={adquisición, procesamiento y clasificación (APC)}
}

\newglossaryentry{labview}{
    type=\acronymtype,
    name={LabVIEW},
    description={\emph{Laboratory Virtual Instrument Engineering Workbench}},
    first={Laboratory Virtual Instrument Engineering Workbench (LabVIEW\textregistered)}
}

\newglossaryentry{ssvep}{
    type=\acronymtype,
    name={SSVEP},
    description={Potencial visual evocado de estado estacionario (\emph{steady state visually evoked potential})},
    first={laboratorio de instrumentación virtual (LabVIEW\textregistered)}

    name={SSVEP},
    description={Potencial visual evocado de estado estacionario (\emph{steady state visually evoked potential})},
    first={potencial visual evocado de estado estacionario (SSVEP, \emph{steady state visually evoked potential})},
    plural={SSVEP},
    descriptionplural={potenciales evocados visual de estado estacionario},
    firstplural={potenciales evocados visual de estado estacionario (SSVEP, \emph{steady state visually evoked potentials})},
}

\newglossaryentry{dsp}{
    type=\acronymtype,
    name={DSP},
    description={Procesamiento digital de señales (\emph{digital signal processing})},
    first={procesamiento digital de señales (DSP, \emph{digital signal processing})}
}

\newglossaryentry{ai}{
    type=\acronymtype,
    name={AI},
    description={Inteligencia artificial (\emph{artificial intelligent})},
    first={inteligencia artificial (AI, \emph{artificial intelligent})},
}

\newglossaryentry{bp}{
    type=\acronymtype,
    name={BP},
    description={Retro-propagación (\emph{backpropagation})},
    first={retro-propagación (BP, \emph{backpropagation})},
}

\newglossaryentry{gd}{
    type=\acronymtype,
    name={GD},
    description={Gradiente descendente (\emph{gradient descent})},
    plural={GD},
    longplural={descensos del gradiente},
    first={gradiente descendente (GD, \emph{gradient descent})},
}

\newglossaryentry{rmse}{
    type=\acronymtype,
    name={RMSE},
    description={Raíz cuadrada del error cuadrático medio (\emph{root mean square error})},
    first={raíz cuadrada del error cuadrático medio (RMSE, \emph{root mean square error})}
}

\newglossaryentry{tvblf}{
    type=\acronymtype,
    name={TVBLF},
    description={Barrera de Lyapunov variante en el tiempo (\emph{time-variant barrier Lyapunov function})},
    first={barrera de Lyapunov variante en el tiempo (TVBLF, \emph{time-variant barrier Lyapunov function})},
}

\newglossaryentry{blf}{
    type=\acronymtype,
    name={BLF},
    description={Barrera de Lyapunov (\emph{barrier Lyapunov function})},
    first={barrera de Lyapunov (BLF, \emph{barrier Lyapunov function})},
}

\newglossaryentry{smc}{
    type=\acronymtype,
    name={SMC},
    description={Control en modos deslizantes (\emph{sliding mode control})},
    first={control en modos deslizantes (SMC, \emph{sliding mode control})},
}