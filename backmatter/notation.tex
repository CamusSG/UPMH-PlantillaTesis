\newglossaryentry{not:defined as}{
    type=symbols,
    parent=notation,
    name=\ensuremath{\overset{\text{def}}{=}},
    description={definido como},
    sort=1
}

\newglossaryentry{not:approximatelly equal}{
    type=symbols,
    parent=notation,
    name=\ensuremath{\approx},
    description={aproximadamente igual a},
    sort=2
}

\newglossaryentry{not:identically equal}{
    type=symbols,
    parent=notation,
    name=\ensuremath{\equiv},
    description={equivaliente a},
    sort=3
}

\newglossaryentry{not:not equals to}{
    type=symbols,
    parent=notation,
    name=\ensuremath{\neq},
    description={diferente a},
    sort=4
}

\newglossaryentry{not:less and greater than}{
    type=symbols,
    parent=notation,
    name=\ensuremath{<~(>)},
    description={menor que (mayor que)},
    sort=5
}

\newglossaryentry{not:less and greater than or equal to}{
    type=symbols,
    parent=notation,
    name=\ensuremath{\leq~(\geq)},
    description={menor o igual que (mayor o igual que)},
    sort=6
}

\newglossaryentry{not:for all}{
    type=symbols,
    parent=notation,
    name=\ensuremath{\forall},
    description={para todo},
    sort=10
}

\newglossaryentry{not:belongs of}{
    type=symbols,
    parent=notation,
    name=\ensuremath{\in},
    description={pertenece a},
    sort=11
}

\newglossaryentry{not:subset of}{
    type=symbols,
    parent=notation,
    name=\ensuremath{\subset},
    description={subconjunto de},
    sort=12
}

\newglossaryentry{not:union intersection}{
    type=symbols,
    parent=notation,
    name=\ensuremath{\cup~(\cap)},
    description={unión de conjuntos (intersección de conjuntos)},
    sort=13
}

\newglossaryentry{not:tends to}{
    type=symbols,
    parent=notation,
    name=\ensuremath{\to},
    description={tiende a},
    sort=14
}

\newglossaryentry{not:implies}{
    type=symbols,
    parent=notation,
    name=\ensuremath{\Rightarrow},
    description={implica que},
    sort=15
}

\newglossaryentry{not:if and only if}{
    type=symbols,
    parent=notation,
    name=\ensuremath{\Longleftrightarrow},
    description={si y sólo si},
    sort=16
}

\newglossaryentry{not:and or}{
    type=symbols,
    parent=notation,
    name=\ensuremath{\wedge~(\vee)},
    description={conjunción <<y>> (conjunción <<o>>)},
    sort=17
}

\newglossaryentry{not:sumatory}{
    type=symbols,
    parent=notation,
    name=\ensuremath{\sum},
    description={sumatoria},
    sort=20
}

\newglossaryentry{not:product}{
    type=symbols,
    parent=notation,
    name=\ensuremath{\prod},
    description={producto},
    sort=21
}

\newglossaryentry{not:absolute value}{
    type=symbols,
    parent=notation,
    name=\ensuremath{|a|},
    description={valor absoluto de un escalar $a$},
    sort=25
}

\newglossaryentry{not:norm of vector}{
    type=symbols,
    parent=notation,
    name=\ensuremath{\|\boldsymbol{x}\|^2},
    description={norma L2 de un vector $\boldsymbol{x}$},
    sort=26
}

\newglossaryentry{not:gradient}{
    type=symbols,
    parent=notation,
    name=\ensuremath{\nabla f},
    description={vector gradiente},
    sort=30
}

\newglossaryentry{not:partial derivation}{
    type=symbols,
    parent=notation,
    name=\ensuremath{\dfrac{\partial f}{\partial t}},
    description={derivada parcial respecto al tiempo},
    sort=31
}

\newglossaryentry{not:transpose}{
    type=symbols,
    parent=notation,
    name=\ensuremath{\boldsymbol{A}^\top~(\boldsymbol{x}^\top)},
    description={matriz transpuesta (vector transpuesto)},
    sort=40
}

% \newglossaryentry{not:euler}{
%     type=symbols,
%     parent=notation,
%     name=\ensuremath{e},
%     description={número de Euler},
%     sort=50
% }

% \newglossaryentry{not:pi}{
%     type=symbols,
%     parent=notation,
%     name=\ensuremath{\pi},
%     description={valor geométrico},
%     sort=51
% }

\newglossaryentry{not:real number}{
    type=symbols,
    parent=notation,
    name=\ensuremath{\mathbb{R}},
    description={conjunto de números reales},
    sort=60
}

\newglossaryentry{not:integer number}{
    type=symbols,
    parent=notation,
    name=\ensuremath{\mathbb{Z}},
    description={conjunto de números enteros},
    sort=61
}

\newglossaryentry{not:exponential}{
    type=symbols,
    parent=notation,
    name=\ensuremath{\operatorname{exp}(\cdot)},
    description={función exponencial},
    sort=65
}

\newglossaryentry{not:generalized cosine windows}{
    type=symbols,
    parent=notation,
    name=\ensuremath{\operatorname{gcs}(\cdot)},
    description={ventanas de coseno generalizadas},
    sort=66
}

% \newglossaryentry{not:imaginary}{
%     type=symbols,
%     parent=notation,
%     name=\ensuremath{i},
%     description={número imaginario},
%     sort=62
% }