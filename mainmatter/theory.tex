\initchapter{Marco teórico}\label{chap:theory}

\lipsum[17]

\section{Tema}\label{sec:theory bci}

    \glsaddall

    Para hacer referencias cruzadas usar
    \begin{lstlisting}
        \cref{label} % Consultar \usepackage{cleveref}
    \end{lstlisting}
    en vez de
    \begin{lstlisting}
        \ref{label}
    \end{lstlisting}

    En la \cref{fig:enter-label}... 
    
    \begin{figure}[htb]
        \centering
        \includegraphics[scale=1.2]{graphics/titlepage/mia.jpg}
        \caption{Caption}
        \label{fig:enter-label}
    \end{figure}

    El área del círculo está dada por la \cref{eq:area} donde $r$ indica el radio.
    
    \begin{equation}
        \label{eq:area}
        A \gls{not:defined as} \pi r^2
    \end{equation}

\subsection{Subtema}