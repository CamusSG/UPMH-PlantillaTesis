\initchapter{Introducción}\label{chap:intro}

\section{Planteamiento del problema}

    \lipsum[17]

\section{Descripción de la solución propuesta}

    \lipsum[17]

\section{Objetivos de la tesis}

    \subsection*{Objetivo general}
         
    \subsection*{Objetivos específicos}

        \begin{itemize}
            \item Elemento 1
            \item Elemento 2...
        \end{itemize}
    
\section{Justificación}

    \lipsum[17]

\section{Hipótesis}
    
    \lipsum[17]

\section{Metodología}

\section{Alcances}
    
\section{Limitaciones}

\section{Aportaciones}

\section{Organización de la tesis}

    El \textit{Estado del arte} se encuentra en el capítulo \ref{chap:state art}, donde se analizan los trabajos relacionados con los temas de investigación de esta tesis. En el capítulo \ref{chap:theory} se presenta el \textit{Marco teórico} de las áreas de conocimiento involucradas. Enseguida, se tratarán el \textit{Diseño del controlador proporcional multiresolución} y el \textit{Diseño de la interfaz cerebro-computadora} en los capítulos \ref{chap:pmr ctrl} y \ref{chap:design bci}, respectivamente. Finalmente, se tienen las \textit{Conclusiones} en el capítulo \ref{chap:concl}, donde se exponen algunas recomendaciones para el trabajo a futuro.